\documentclass[10pt]{article}

\usepackage{fullpage}
\usepackage{setspace}
\usepackage{parskip}
\usepackage{titlesec}
\usepackage{xcolor}
\usepackage{lineno}





\PassOptionsToPackage{hyphens}{url}
\usepackage[colorlinks = true,
            linkcolor = blue,
            urlcolor  = blue,
            citecolor = blue,
            anchorcolor = blue]{hyperref}
\usepackage{etoolbox}
\makeatletter
\patchcmd\@combinedblfloats{\box\@outputbox}{\unvbox\@outputbox}{}{%
  \errmessage{\noexpand\@combinedblfloats could not be patched}%
}%
\makeatother


\usepackage[round]{natbib}
\let\cite\citep




\renewenvironment{abstract}
  {{\bfseries\noindent{\abstractname}\par\nobreak}\footnotesize}
  {\bigskip}

\renewenvironment{quote}
  {\begin{tabular}{|p{13cm}}}
  {\end{tabular}}

\titlespacing{\section}{0pt}{*3}{*1}
\titlespacing{\subsection}{0pt}{*2}{*0.5}
\titlespacing{\subsubsection}{0pt}{*1.5}{0pt}


\usepackage{authblk}


\usepackage{graphicx}
\usepackage[space]{grffile}
\usepackage{latexsym}
\usepackage{textcomp}
\usepackage{longtable}
\usepackage{tabulary}
\usepackage{booktabs,array,multirow}
\usepackage{amsfonts,amsmath,amssymb}
\providecommand\citet{\cite}
\providecommand\citep{\cite}
\providecommand\citealt{\cite}
% You can conditionalize code for latexml or normal latex using this.
\newif\iflatexml\latexmlfalse
\providecommand{\tightlist}{\setlength{\itemsep}{0pt}\setlength{\parskip}{0pt}}%

\AtBeginDocument{\DeclareGraphicsExtensions{.pdf,.PDF,.eps,.EPS,.png,.PNG,.tif,.TIF,.jpg,.JPG,.jpeg,.JPEG}}

\usepackage[utf8]{inputenc}
\usepackage[T2A]{fontenc}
\usepackage[russian,ngerman,english]{babel}








\begin{document}

\title{}



\author[1]{Vasili Shymanski}%
\affil[1]{Affiliation not available}%


\vspace{-1em}



  \date{\today}


\begingroup
\let\center\flushleft
\let\endcenter\endflushleft
\maketitle
\endgroup









\selectlanguage{russian}Руководство по Scrum@Scale {}

~

Наиболее полное руководтво по Scrum@Scale.

Маштабирование которое Работает

~

Версия 1,04 -- 17 января 2019 года

\par\null\par\null

Цель Руководства по Scrum@Scale{}

Скрам, изначально изложеный в Руководстве по Скраму, является
фреймворком для разработки, поставки и поддержки сложных продуктов одной
коммандой. С начала существования, его использование расширилось для
разработки продуктов, процессов, услуг, и систем требующих усилий
нескольких комманд. Scrum@Scale был создан для эффективной координации
этой новой экосистемы комманд оптемизируя ~общую стратегию организации.
Эта цель достигается основанием \selectlanguage{english},,minimal'no zhizniesposobnoi biurokratii''
ispol'zuiuia svobodnuiu dlia mashtabirovniia arkhittiekturu, kotoraia rasshiriaietsia
putiem obychnoi skram komandy funktsioniruiushchiei v ramkakh tsieloi orghanizatsii.

~

Eto rukovodstvo sotoit iz opriedielienii chastiei kotoryie sostavliaiut
Scrum@Scale freimvork, vkliuchaia v siebia mashtabirovannyie roli,
mashtabirovannyie sobytiia, artifakty priedpriiatiia, i pravila kotoryie
obiediniaiut ikh vmiestie.

~

Dr. Jeff Sutherland razrabotal Scrum@Scale baziruias' na fundamiental'nykh
printsipakh osnovannykh na Skramie, slozhnaia adaptativnoi tieorii, tieorii ighr,
i obiektno-oriientirovannoi tiekhnologhii. Eto rukovodstvobylo razrabotanno
pri uchastii mnoghikh opytnykh Skram praktikov osnovyvaias' na riezul'tatakh v
ikh fierie dieiatiel'nosti. Tsiel' etogho rukovodstva priedostavit' chitatieliu
vozmozhnost' impliemientatsii Scrum@Scale samostoiatiel'no.

~

Pochiemu Scrum@Scale?

Skram byl razrabotan dlia odnoi komandy, chtoby imiet' vozmozhnost' rabotat'
rabotat' v optimal'nom riezhimie sokhraniaia ustoichivaie razvitiie. Na praktikie
bylo obnaruzhieno chto pri umielichiennii kolichiestva Skram komand v ramkakh
rosta orghanizatsii, vykhod (rabochii produkt) i skorost' etikh komand
nachinaiet padat' (iz-za takikh prichin kak kross komandnyie zavisimosti i
dublirovaniie raboty). Blaghodoria etomu stalo ochievidno, chto triebuietsia
freimvork dlia efiektivnoi koordinatsii miezhdu etimi komandami dlia
dostizhieniia linieinogho mashtabirovaniia.

~

Pri ispol'zovanii svobodno-mashtabiroiemoi arkhitiektury, orghanizatsiia nie
oghranichiena k rostu v opriedieliennyi sposob chieriez kompliekt~ arbitral'nykh
pravil; skorieie ona mozhiet rasti natural'nym putiem osnovyvaias' na svoikh
unikal'nykh potriebnostiakh i ustoichivom tiempie izmienienii, kotoryi mozhiet byt'
priniat ghruppami liudiei, sostavliaiushchimi orghanizatsiiu. Prostota modieli
Scum@Scale nieobkhodima dlia svobodno-mashtabiruieboi arkhitiektury~ i
ostorozhno izbieghaiet dobavlieniia dopolnitiel'noi slozhnosti, kotoraia budiet
prichinoi padienia produktivnosti komand po mierie uvieliechieniia ikh kolichiestva.

~

Scum@Scale priednaznachien dlia mashtabirovaniia orghanizatsii tsielikom: vsiekh
otdielov, produktov i uslugh. On mozhien byt' priemienien vo vsiekh oblastiakh vshch
vsiekh tipakh orghanizatsii v industrii, pravidiel'stvie, obrazovanii.

Opriedielieniie Scrum@Scale

Skram: Frieimvork v ramkakh kotorogho liudi moghut rieshat' slozhnyie
adaptativnyie probliemy, produktivno i krieativno dostavliaia produkty
naivysshiei tsiennosti.

Skram ghaid priedstavliaiet soboi minimal'nyi nabor osobiennostiei pozvoliaiushchii
ispiektirovaniie i adaptirovaniie chieriez radikal'nuiu prozrachnost',
privodiashchii k innovatsiinosti, udovlietvoniiu kliienta, proizvoditiel'nosti i
schast'iu komandy.

Scrum@Scale: freimvork v ramkhakh kotorogho sieti Skram-kommand postanno
rabotaiushchiie v ramkakh Skram ghaida~ moghut adriesovat' slozhno-adaptativnyie
probliemyb krieativno dostavlia produkty naivysshiei vosmozhnoi tsiennosti.

~Vnimaniie: Etimi produktami mozhiet byt' apparatura, proghramnoie
obiespiechieniie, slozhno intieghrirovanyie sistiemy, protsiessy libo siervisy i
t.d., v zavisimosti ot togho v kakoi sfierie rabotaiut Skram komandy.

Scrum@Scale:

\selectlanguage{ngerman}·~~~~~~~~ \selectlanguage{russian}Легкий -- минимальный уровень бюрократии {}

\selectlanguage{ngerman}·~~~~~~~~ \selectlanguage{russian}Легкий к пониманию -- состоящий только из Скрам команд

\selectlanguage{ngerman}·~~~~~~~~ \selectlanguage{russian}Сложный в достижении совершенства -- нуждается в внедрении
новой рабочей модели

Scrum@Scale это фрэйворк для маштабирования Скрама. Он значительно
упрощает маштабирование скрама используя скрам.

В скраме, уделени внимание разденению ответсвенности между \selectlanguage{english},,Chto'' i
\selectlanguage{english},,Kak''. Takoie zhie vnianiie udielieno i v Scrum@Scale, takim obrazom sfiera
polnomochii i otvietsviennosti chietko poniatna chtoby limitirovat' nienuzhnyie
orghanizatsionnyie konflikty, kotoryie mieshaiut komandam v dostizhienii ikh
optimal'noi produktivnosti.

Scrum@Scale sostavlien iz chastiei kotoryie pozvoliaiut orghanizatsiiam
podstraivat' stratieghiiu tranformatsii i iegho primienieniia. Eto daiet im
vozmozhnost' opriedielit' tsieli dlia inkriemiendal'noi prioritiezatsii usilii po
izmienieniiu v toi oblasti, kotoruiu oni schitaiut naibolieie tsiennoi ili
naibolieie nuzhdaiushchieisia v izmienieniiakh, a zatiem pieriekhodiat k drughim.

V razdielienii etikh dvukh {}

~

~

~

\selectlanguage{english}
\clearpage
\end{document}

